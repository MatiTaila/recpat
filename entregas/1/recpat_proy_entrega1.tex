%\documentclass[spanish,12pt,a4paper,titlepage,twoside,openright]{scrbook}
\documentclass[12pt,a4paper,titlepage]{report}
%\usepackage[latin1]{inputenc}
\usepackage[utf8]{inputenc}
\usepackage{graphicx}
\usepackage{subfig}
\usepackage{float}
\usepackage{wrapfig}
\usepackage{multirow}
\usepackage{caption}
\usepackage[spanish,es-nodecimaldot]{babel}
\usepackage[dvips]{hyperref}
\usepackage{amssymb}
\usepackage{listings}
\usepackage{epsfig}
\usepackage{amsmath}
\usepackage{array}
\usepackage[table]{xcolor}
\usepackage{multirow}
\usepackage{hhline}
\usepackage{cancel}

\usepackage[Sonny]{fncychap}
%\usepackage[Glenn]{fncychap}
%\usepackage[Conny]{fncychap}
%\usepackage[Rejne]{fncychap}
%\usepackage[Bjarne]{fncychap}

\usepackage{subfiles}
\usepackage{framed}
\usepackage{appendix}
\setlength{\topmargin}{-1.5cm}
\setlength{\textheight}{25cm}
\setlength{\oddsidemargin}{0.3cm} 
\setlength{\textwidth}{15cm}
\setlength{\columnsep}{0cm}
%\setkomafont{disposition}{\normalfont\bfseries}
\captionsetup{tablename=Tabla}

\ChNameVar{\bfseries\LARGE\sf}\ChNumVar{\fontsize{62}{65}\selectfont}
\ChTitleVar{\bfseries\LARGE\sf} \ChRuleWidth{2pt} \ChNameAsIs
\ChTitleAsIs
\renewcommand\FmN[4]{}
\newcommand{\HRule}{\rule{\linewidth}{0.5mm}}
\newcommand{\bs}{\boldsymbol}

\begin{document}


\begin{titlepage}
\begin{center}
\vspace{2cm}
\textsc{\LARGE Facultad de Ingenier\'ia de la Universidad de la Rep\'ublica}\\[1.5cm]
\vspace{2cm}
\textsc{\Large Introducción al Reconocimiento de Patrones  \\[1cm]  Curso 2013}\\[0.5cm]
\vspace{2cm}
% Title
\HRule \\[0.4cm]
{ \huge \bfseries Entrega 1}\\[0.4cm]
\HRule \\[1.5cm]
\vspace{2cm}
% Author and supervisor
\begin{minipage}{0.4\textwidth}
\begin{flushleft} \large
\emph{Autores:}\\
José Luis \textsc{Nunes}\\
Leonardo \textsc{Pujadas} \\
Mat\'ias \textsc{Tailani\'an}
\end{flushleft}
\end{minipage}
\begin{minipage}{0.4\textwidth}
\begin{flushright} \large
\end{flushright}
\end{minipage}

\vspace{3cm}

\vfill
\begin{figure} [h!]
\centering
\subfloat{\includegraphics[width=0.25\textwidth]{./pics/logoIIE_transparente.png}}\hspace{1cm}
\subfloat{\includegraphics[width=0.15\textwidth]{./pics/logo_fing_transparente.png}}\hspace{1cm}
\subfloat{\includegraphics[width=0.15\textwidth]{./pics/logo_udelar.png}}
\end{figure}

% Bottom of the page
{\large \today}
\end{center}
\end{titlepage}

\chapter{Descripción del problema}

El problema se enmarca dentro del proyecto \emph{``Incorporación de datos imagenológicos a las bases de datos fenotípicas de bovinos para la identificación de genes significativos para mejorar las características reproductivas y de calidad de carne''}, del que formamos parte.\\

El problema que se atacará en este proyecto es un problema de gran interés en la industria cárnica, ya que una mejora genética del ganado producirá un aumento en la calidad de la carne, que repercutirá directamente en la economía del país.

\section*{Objetivo}

El objetivo principal del proyecto es la investigación en técnicas que permitan
contribuir con la predicción de fertilidad de rodeo lechero y la calidad de la carne
integrando técnicas de reconocimiento de patrones sobre datos de alta dimensión.\\

Más concretamente se estudiará la correlación entre datos fenotípicos y genotípicos de ganado para de esta manera identificar cuáles son los genes más significativos que afectan directamente en la calidad de carne y las características reproductivas.

\section*{Enfoque}

Las técnicas y algoritmos que se utilizarán no están del todo definidos. El estimador que se utilizará como base de referencia es el de Máxima Verosimilitud Restringida (\emph{REML}), que es el estimador clásico que usan los genetistas.\\

Las características fenotípicas a priori más relevantes que se intentarán correlacionar con los datos genotípicos son las el área del ojo de bife, el porcentaje de grasa intramuscular y el peso. Estas características son las que están más relacionadas con la calidad de la carne. Se cuenta con más de 40 características por lo cual será necesario realizar una etapa de selección de las características más relevantes.

\section*{Alcance}

El alcance del proyecto pretende realizar una primera aproximación a las técnicas de correlacionado de datos de alta dimensión, apuntando a la obtención de un resultado preliminar sobre cuáles son los genes más significativos para la mejora de las características reproductivas y la calidad de la carne.

\section*{Datos}
A lo largo de los últimos años varios grupos de Facultad de Veterinaria han creado bases de datos con información fenotípica y molecular relacionada con la calidad cárnica y la fertilidad bovina aplicada a la producción lechera. Este análisis requiere el relevamiento masivo de datos y su procesamiento en forma exhaustiva y eficiente. \\

Se tiene disponible una base de datos de alrededor de 700 individuos, donde para cada uno de ellos se tienen en el entorno de 45 características. Una de las dificultades que presenta este trabajo es que los tipos de datos de las características son muy variados. Algunos de los datos son cuantitativos y varían en diferentes escalas de valores, mientras que otros son cualitativos. A su vez se está analizando la posibilidad de incluir otra base de datos de similares características, pero de vacas lecheras.
\section*{Cronograma}

Se planea realizar las siguientes actividades para las instancias:
\begin{itemize}
	\item \emph{Entrega de avance intermedio}
	\begin{itemize}
		\item Obtención y puesta a punto de las bases de datos
		\item Estudio del estado del arte en técnicas de reconocimiento de patrones en alta dimensión.
	\end{itemize}
	\item \emph{Entrega de reporte final}
	\begin{itemize}
		\item Implementación
	\end{itemize}
\end{itemize}

\begin{thebibliography}{99}

\bibitem{bib:Eileen}Eileen Armstrong Reborati: Detección y análisis de genes asociados a la calidadd e carne en bovinos, En \emph{Tesis de doctorado}, Madrid 2011.

\bibitem{bib:paper_rico}Jennifer L. Gill, Stephen C. Bishop, Caroline McCorquodale, John L. Williams, Pamela Wiener: Associations between single nucleotide polymorphisms in multiple candidate genes and carcass and meat quality traits in a commercial Angus-cross population, En \emph{Elsevier}, July 2010.

\end{thebibliography}


%la búsqueda de marcadores moleculares para el mejoramiento genético del ganado.

%interrelación de estas variables productivas y reproductivas

\end{document}